
\documentclass[10pt,landscape]{article}
\usepackage{amssymb,amsmath,amsthm,amsfonts}
\usepackage{multicol,multirow}
\usepackage{calc}
\usepackage{ifthen}
\usepackage[landscape]{geometry}
\usepackage[spanish,activeacute]{babel}
\usepackage[colorlinks=true,citecolor=blue,linkcolor=blue]{hyperref}


\ifthenelse{\lengthtest { \paperwidth = 11in}}
    { \geometry{top=.5in,left=.5in,right=.5in,bottom=.5in} }
	{\ifthenelse{ \lengthtest{ \paperwidth = 297mm}}
		{\geometry{top=1cm,left=1cm,right=1cm,bottom=1cm} }
		{\geometry{top=1cm,left=1cm,right=1cm,bottom=1cm} }
	}
\pagestyle{empty}
\makeatletter
\renewcommand{\section}{\@startsection{section}{1}{0mm}%
                                {-1ex plus -.5ex minus -.2ex}%
                                {0.5ex plus .2ex}%x
                                {\normalfont\large\bfseries}}
\renewcommand{\subsection}{\@startsection{subsection}{2}{0mm}%
                                {-1explus -.5ex minus -.2ex}%
                                {0.5ex plus .2ex}%
                                {\normalfont\normalsize\bfseries}}
\renewcommand{\subsubsection}{\@startsection{subsubsection}{3}{0mm}%
                                {-1ex plus -.5ex minus -.2ex}%
                                {1ex plus .2ex}%
                                {\normalfont\small\bfseries}}
\makeatother
\setcounter{secnumdepth}{0}
\setlength{\parindent}{0pt}
\setlength{\parskip}{0pt plus 0.5ex}
% -----------------------------------------------------------------------

\title{Quick Guide to clustalw tool.}

\begin{document}

\raggedright
\footnotesize

\begin{center}
     \Large{\textbf{ClustalW}} \\
\end{center}
\begin{multicols}{3}
\setlength{\premulticols}{1pt}
\setlength{\postmulticols}{1pt}
\setlength{\multicolsep}{1pt}
\setlength{\columnsep}{2pt}

\section{Qu'e es ClustalW? }
BLAST (Basic Local Alignment Search Tool) es un programa bioinformático de
alineamiento de secuencias biol'ogicas, ya sea de ADN, ARN o de prote'inas,
comparando una secuencia problema con m'ultiples secuencias dentro de una base
de datos. 

\subsection{Alineamiento m'ultiple con un archivo de secuencias.}
\textbf{clustalw} - Herramienta para alineamiento m'ultiple de secuencias.
  \begin{enumerate}

    \item \textbf{Ejecutar el alineamiento m'ultiple con clustalw}
      \begin{verbatim}
        clustalw -infile=input.fasta -seqnos=ON
 -gapopen=2 -gapext=0.5
      \end{verbatim}
  \end{enumerate}


\subsection{Alineamiento m'ultiple interactivo.}
\begin{enumerate}
  \item Ejecutar desde la terminal \textbf{clustalw}
  \item Seleccionar la opci'on 1 "Sequence Input From Disc" y especificar el
    nombre del archivo de entrada.
  \item Seleccionar la opcion 2 "Multiple alignments".
\item Seleccionar la opci'on 9 "Output format options", a continuaci'on
  seleccionar la opci'on 3 "Toggle GCG/MSF format output" para guardar la salida
  en formato "MSF".
\item Presionar la tecla Enter para regresar al menu anterior.
\item Seleccionar la opci'on 1 "Do complete multiple alignment now
  (Slow/Accurate)"
\end{enumerate}

\section{Recursos}
Great symbol look-up site:
Bash commands Reference\href{https://dev.to/awwsmm/101-bash-commands-and-tips-for-beginners-to-experts-30je}{Dev
Journal 101 Bash commands for beginners to experts}\\
\vfill
\hrule
~\\
Jos'e Manuel S, \href{http://github.com/J0MS}{http://github.com/J0MS/}
\end{multicols}

\end{document}
