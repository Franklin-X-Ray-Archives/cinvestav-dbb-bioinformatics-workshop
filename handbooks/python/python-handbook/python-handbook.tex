
\documentclass[10pt,landscape]{article}
\usepackage{amssymb,amsmath,amsthm,amsfonts}
\usepackage{multicol,multirow}
\usepackage{calc}
\usepackage{ifthen}
\usepackage[landscape]{geometry}
\usepackage[spanish,activeacute]{babel}
\usepackage[colorlinks=true,citecolor=blue,linkcolor=blue]{hyperref}
\usepackage{graphicx}

\ifthenelse{\lengthtest { \paperwidth = 11in}}
    { \geometry{top=.5in,left=.5in,right=.5in,bottom=.5in} }
	{\ifthenelse{ \lengthtest{ \paperwidth = 297mm}}
		{\geometry{top=1cm,left=1cm,right=1cm,bottom=1cm} }
		{\geometry{top=1cm,left=1cm,right=1cm,bottom=1cm} }
	}
\pagestyle{empty}
\makeatletter
\renewcommand{\section}{\@startsection{section}{1}{0mm}%
                                {-1ex plus -.5ex minus -.2ex}%
                                {0.5ex plus .2ex}%x
                                {\normalfont\large\bfseries}}
\renewcommand{\subsection}{\@startsection{subsection}{2}{0mm}%
                                {-1explus -.5ex minus -.2ex}%
                                {0.5ex plus .2ex}%
                                {\normalfont\normalsize\bfseries}}
\renewcommand{\subsubsection}{\@startsection{subsubsection}{3}{0mm}%
                                {-1ex plus -.5ex minus -.2ex}%
                                {1ex plus .2ex}%
                                {\normalfont\small\bfseries}}
\makeatother
\setcounter{secnumdepth}{0}
\setlength{\parindent}{0pt}
\setlength{\parskip}{0pt plus 0.5ex}
% -----------------------------------------------------------------------

\title{Quick Guide to Linux python.}

\begin{document}

\raggedright
\footnotesize

\begin{center}
     \Large{\textbf{Python}} \\
\end{center}
\begin{multicols}{3}
\setlength{\premulticols}{1pt}
\setlength{\postmulticols}{1pt}
\setlength{\multicolsep}{1pt}
\setlength{\columnsep}{2pt}

\section{Qu'e es Python? }
Python es un lenguaje interpretado, de tipado din'amico, multiparadigma, de alto nivel
desarrollado y mantenido por la Python Software Foundation.\\
  \includegraphics[width=0.075\textwidth]{python-logo}\\
\section{Instalar int'erprete y paquetes de Python.}
Instalar el int'erprete de Python en sistemas operativos derivados de Ubuntu
(Ej. Biolinux):\vspace{0.5cm}\\
Instalar python3 \textbf{sudo apt-get install python3}\\
Instalar python2 \textbf{sudo apt-get install python}
\vspace{0.5cm}\\
Instalar la herramienta de instalaci'on de paquetes de Python en sistemas operativos GNU/Linux:
Instalar pip3 para paquetes en python3 con: \textbf{sudo apt-get install python3-pip}\\
Instalar pip para paquetes en python con: \textbf{sudo apt-get install python-pip}



\section{Recursos}
Python Software Foundation: \href{http://www.python.org/}{Python Official Site}\\
Python Official Documentation \href{https://docs.python.org/3/ }{Python Documentation}\\
Python Package Index\href{https://pypi.org/}{PyPI}\\ 
\vfill
\hrule
~\\
Jose Manuel S, \href{http://github.com/J0MS}{http://github.com/J0MS}
\end{multicols}

\end{document}
