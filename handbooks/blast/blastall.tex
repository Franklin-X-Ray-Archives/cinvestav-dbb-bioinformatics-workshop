
\documentclass[10pt,landscape]{article}
\usepackage{amssymb,amsmath,amsthm,amsfonts}
\usepackage{multicol,multirow}
\usepackage{calc}
\usepackage{ifthen}
\usepackage[landscape]{geometry}
\usepackage[spanish,activeacute]{babel}
\usepackage[colorlinks=true,citecolor=blue,linkcolor=blue]{hyperref}


\ifthenelse{\lengthtest { \paperwidth = 11in}}
    { \geometry{top=.5in,left=.5in,right=.5in,bottom=.5in} }
	{\ifthenelse{ \lengthtest{ \paperwidth = 297mm}}
		{\geometry{top=1cm,left=1cm,right=1cm,bottom=1cm} }
		{\geometry{top=1cm,left=1cm,right=1cm,bottom=1cm} }
	}
\pagestyle{empty}
\makeatletter
\renewcommand{\section}{\@startsection{section}{1}{0mm}%
                                {-1ex plus -.5ex minus -.2ex}%
                                {0.5ex plus .2ex}%x
                                {\normalfont\large\bfseries}}
\renewcommand{\subsection}{\@startsection{subsection}{2}{0mm}%
                                {-1explus -.5ex minus -.2ex}%
                                {0.5ex plus .2ex}%
                                {\normalfont\normalsize\bfseries}}
\renewcommand{\subsubsection}{\@startsection{subsubsection}{3}{0mm}%
                                {-1ex plus -.5ex minus -.2ex}%
                                {1ex plus .2ex}%
                                {\normalfont\small\bfseries}}
\makeatother
\setcounter{secnumdepth}{0}
\setlength{\parindent}{0pt}
\setlength{\parskip}{0pt plus 0.5ex}
% -----------------------------------------------------------------------

\title{Quick Guide to Blastall tool.}

\begin{document}

\raggedright
\footnotesize

\begin{center}
     \Large{\textbf{BLAST.Basic Local Alignment Search Tool}} \\
\end{center}
\begin{multicols}{3}
\setlength{\premulticols}{1pt}
\setlength{\postmulticols}{1pt}
\setlength{\multicolsep}{1pt}
\setlength{\columnsep}{2pt}

\section{Qu'e es BLAST? }
BLAST (Basic Local Alignment Search Tool) es un programa bioinformático de
alineamiento de secuencias biol'ogicas, ya sea de ADN, ARN o de prote'inas,
comparando una secuencia problema con m'ultiples secuencias dentro de una base
de datos. 

\subsection{blastn.}
\textbf{blastn} - Herramienta para alineamiento de secuencias de nucle'otidos
  \begin{enumerate}
    \item \textbf{Descargar la base de datos de nucle'otidos}\\
      \href{ftp://ftp.ncbi.nih.gov/blast/db/FASTA/yeast.nt.gz}{wget
        ftp://ftp.ncbi.nih.gov/blast/db/FASTA/yeast.nt.gz}
    \item \textbf{Descomprimir el archivo *.nt.gz} \\
      \begin{verbatim}
  gunzip yeast.nt.gz
  \end{verbatim}

    \item \textbf{Formatear la base de datos con formatdb}
      \begin{verbatim}
  formatdb -i yeast.nt -p F -o T
      \end{verbatim}


    \item \textbf{Ejecutar el alineamiento con blastn}
      \begin{verbatim}
  blastall -p blastn -d yeast.nt -i input.fasta 
-o result.out
      \end{verbatim}
  \end{enumerate}

\subsection{blastp.}
\textbf{blastn} - Herramienta para alineamiento de secuencias de amino'acidos.
  \begin{enumerate}
    \item \textbf{Descargar la base de datos de amino'acidos}\\
      \href{ftp://ftp.ncbi.nih.gov/blast/db/FASTA/yeast.aa.gz}{wget
        ftp://ftp.ncbi.nih.gov/blast/db/FASTA/yeast.aa.gz}
    \item \textbf{Descomprimir el archivo *.aa.gz} \\
      \begin{verbatim}
  gunzip yeast.aa.gz
  \end{verbatim}

    \item \textbf{Formatear la base de datos con formatdb}
      \begin{verbatim}
  formatdb -i yeast.aa -p T -o T
      \end{verbatim}


    \item \textbf{Ejecutar el alineamiento con blastp}
      \begin{verbatim}
  blastall -p blastp -d yeast.aa -i input.fasta -T T
-o result.out
      \end{verbatim}
  \end{enumerate}

\subsection{blastx.}
\textbf{blastx} - B'usqueda de secuencias de prote'inas a partir de secuencias
de nucle'otidos. 
  \begin{enumerate}

    \item \textbf{A partir del archivo *.nt , formatear la base de datos con formatdb}
      \begin{verbatim}
  formatdb -i yeast.nt -p F -o T
      \end{verbatim}


    \item \textbf{Ejecutar el alineamiento con blastx}
      \begin{verbatim}
  blastall -p blastx -d yeast.nt -i input.fasta -T T 
-o result.html
      \end{verbatim}
  \end{enumerate}


\section{Recursos}
FTP Blast DataBase \href{ftp://ftp.ncbi.nih.gov/blast/db/} {FTP NCBI Database}\\
Documentation for \textbf{formatdb} options
\href{https://www.ncbi.nlm.nih.gov/Class/BLAST/formatdbopts.txt} {Options for
  formatdb tool} 
\vfill
\hrule
~\\
Jos'e Manuel S, \href{http://github.com/J0MS  }{http://github.com/J0MS/}
\end{multicols}

\end{document}
