
\documentclass[10pt,landscape]{article}
\usepackage{amssymb,amsmath,amsthm,amsfonts}
\usepackage{multicol,multirow}
\usepackage{calc}
\usepackage{ifthen}
\usepackage[landscape]{geometry}
\usepackage[spanish,activeacute]{babel}
\usepackage[colorlinks=true,citecolor=blue,linkcolor=blue]{hyperref}


\ifthenelse{\lengthtest { \paperwidth = 11in}}
    { \geometry{top=.5in,left=.5in,right=.5in,bottom=.5in} }
	{\ifthenelse{ \lengthtest{ \paperwidth = 297mm}}
		{\geometry{top=1cm,left=1cm,right=1cm,bottom=1cm} }
		{\geometry{top=1cm,left=1cm,right=1cm,bottom=1cm} }
	}
\pagestyle{empty}
\makeatletter
\renewcommand{\section}{\@startsection{section}{1}{0mm}%
                                {-1ex plus -.5ex minus -.2ex}%
                                {0.5ex plus .2ex}%x
                                {\normalfont\large\bfseries}}
\renewcommand{\subsection}{\@startsection{subsection}{2}{0mm}%
                                {-1explus -.5ex minus -.2ex}%
                                {0.5ex plus .2ex}%
                                {\normalfont\normalsize\bfseries}}
\renewcommand{\subsubsection}{\@startsection{subsubsection}{3}{0mm}%
                                {-1ex plus -.5ex minus -.2ex}%
                                {1ex plus .2ex}%
                                {\normalfont\small\bfseries}}
\makeatother
\setcounter{secnumdepth}{0}
\setlength{\parindent}{0pt}
\setlength{\parskip}{0pt plus 0.5ex}
% -----------------------------------------------------------------------

\title{Quick Guide to Linux Terminal.}

\begin{document}

\raggedright
\footnotesize

\begin{center}
     \Large{\textbf{Terminal}} \\
\end{center}
\begin{multicols}{3}
\setlength{\premulticols}{1pt}
\setlength{\postmulticols}{1pt}
\setlength{\multicolsep}{1pt}
\setlength{\columnsep}{2pt}

\section{Qu'e es un emulador de terminal? }
Un Emulador de terminal es un programa que ejecuta instrucciones para
interactuar con el sistema operativo, realizar tareas administrativas o invocar
aplicaciones locales o en red. Estas instrucciones se escriben en forma de
conjuntos comandos.

\section{Comandos para obtener ayuda atrav'es de la terminal.}
\textbf{man} - Muestra en patalla los manuales de usuario del comando solicitado.
Uso: man comando

\section{Comandos para creaci'on, lectura, localizaci'on y eliminaci'on de archivos.}
\textbf{touch} - Crea un nuevo archivo vacio en el directorio actual de trabajo. 
Uso: touch archivo
\\
\textbf{less} - Lee el contenido de un archivo y lo imprime en pantalla. 
Uso: less archivo
\\
\textbf{find} - Busca un archivo en un directorio.. 
Uso: find ruta archivo
\\
\textbf{cp} - Copia un archivo de una ruta a otra.
Uso: cp origen-del-archivo destino-del-archivo
\\
\textbf{mv} - Mueve un archivo de un directorio a otro.
Uso: mv origen-del-archivo destino-del-archivo
\\
\textbf{rm} - Elimina uno o m'as archivos.
Uso: rm archivo1 archivo2

\section{Comandos para inspeccionar y navegar en un 'arbol de directorios.}
\textbf{mkdir} - Comando para crear un nuevo directorio.
Uso: mkdir nombre-del-directorio
\\
\textbf{dir} - Imprime una lista de directorios dentro del directorio actual de trabajo.
Uso: dir
\\
\textbf{pwd} - Imprime la ruta actual de trabajo
Uso: pwd
\\
\textbf{ls} - Imprime una lista de objetos dentro del directorio actual de trabajo
Uso: ls
\\
\textbf{cd} - Cambiar de directorio.
Uso: cd ruta / cd .. Subir un nivel jer'arquico en el 'arbol de directorios.

\section{Comandos para administrar procesos.}
\textbf{ps} - Comando para obtener una lista de procesos y sus identificadores.
Uso: ps aux | grep nombre-del-programa
\\
\textbf{kill} - Elimina un proceso activo.
Uso: kill SIGNAL ID
\\
killall SIGNAL NOMBRE

\section{Recursos}
Great symbol look-up site:
Bash commands Reference\href{https://dev.to/awwsmm/101-bash-commands-and-tips-for-beginners-to-experts-30je}{Dev
Journal 101 Bash commands for beginners to experts}\\
\vfill
\hrule
~\\
Jos'e Manuel S, \href{http://github.com/J0MS  }{http://github.com/J0MS/}
\end{multicols}

\end{document}
